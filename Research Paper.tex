\documentclass[a4paper,12pt]{article}

% Packages
\usepackage{graphicx} % For including images
\usepackage{amsmath, amssymb} % For mathematical symbols
\usepackage{hyperref} % For hyperlinks
\usepackage{setspace} % For line spacing
\usepackage{cite} % For citations

\begin{document}
% Title and Author
\title{Neuroscience Report Title}
\author{
    Umut Tuna Akgul, 
    Utku Bahcivanoglu, 
    Tommaso Ferracina, 
    George Morris, 
    Tebe Nigrelli  \\  Bocconi University}
\date{\today}


\maketitle

\begin{abstract}
\noindent
This report investigates [brief topic summary]. We explore the methods, results, and implications of our study within the field of neuroscience. 
\end{abstract}

\section{Introduction}
The introduction should provide background information on the topic, including relevant literature and the objectives of the study. 

\section{Methods}

in terms of areas of the brain, we have VISp which is the primary visual cortex which i think we need to include because it’s the “main entry point for visual information” so it provides our baseline reaction. \\

We compare activations with Higher Visual Areas (HVA) so, VISI (lateral visual area), VISal (anterolateral visual area), VISpm (postermodedial VA), and VISam (anteromedial VA)
which are “involved in more complex visual processing”

\section{Results}
Present the findings of the study, including statistical analyses, figures, and tables where applicable.

\section{Discussion}
Interpret the results, discuss their significance, compare them with previous research, and highlight potential limitations and future directions.

\section{Conclusion}
Summarize the key findings and their implications in the field of neuroscience.

\section*{References}
\begin{thebibliography}{99}
\bibitem{example} Author Name, \textit{Title of the Paper}, Journal Name, Volume, Page Numbers, Year.
\end{thebibliography}

\end{document}